
\documentclass[12pt]{exam}
\usepackage[spanish]{babel}
\usepackage{amsmath} % Para notación matemática

\usepackage{amssymb}
\usepackage{multicol} % Para crear columnas
\usepackage[left=10mm, right= 20mm, bottom= 20mm, top= 35mm, headsep=10mm, headheight=0mm]{geometry}
\usepackage{setspace}  
\usepackage{graphicx}
\renewcommand\partlabel{\thepartno)}
%%% DEFINICIONES PARA MODIFICAR APARIENCIA DEL ENCABEZADO%%%
\usepackage{etoolbox}
\makeatletter
\patchcmd{\@fullhead}{\hrule}{\hrule\vskip2pt\hrule height 2pt}{}{}
\patchcmd{\run@fullhead}{\hrule}{\hrule\vskip2pt\hrule height 2pt}{}{}
\makeatother
\pagestyle{headandfoot}
%\runningheadrule
\firstpageheadrule
\firstpageheader {\escuela \\ Curso: \curso}
                 { \\ \tpnombre}
                  {\lugar \\         }       
%\runningheader{\includegraphics[width=0.1\textwidth]{logo.png}}
%{Matemática:Trabajo Práctico}
%{\today}
\firstpagefooter{Prof: \textit{Lucas H.Trejo}}{}{\thepage}
\runningfooter{Prof: \textit{Lucas H.Trejo}}{}{Pag. \thepage\ of \numpages}

%%%%%%%%%%%%%%%%%%%%%%%%%%%%%%%%%%%%%%%%%%%%%%%%%%%%%%%%%%%%%%%%%%%%%%%%%%%%%%%
                                       %%% REEMPLAZAR LOS PARÁMETROS AQUÍ    %%  
\newcommand{\escuela}{EES Nº 32}                                             %%             
\newcommand{\curso}{  5º año}                                                  %%             
\newcommand{\tpnombre}{ COMISIÓN EVALUADORA PREVIOS \\ Matemática 5º año}                     %%
\newcommand{\lugar}{La Plata}                                               %%
%%%%%%%%%%%%%%%%%%%%%%%%%%%%%%%%%%%%%%%%%%%%%%%%%%%%%%%%%%%%%%%%%%%%%%%%%%%%%%%


\begin{document}

\textbf{CONSIGNAS DE EVALUACIÓN:}

\begin{questions}
\question En una sucesión aritmética, $a_1 = 3$ y $r = 5$. Hallar:
    \begin{parts}
        \part El término general $a_n$
        \part Los primeros 10 términos usando el término general.
        \part La suma de los 25 primeros términos de la sucesión.
    \end{parts}

\question Calcular el término pedido:
    \begin{parts}
        \part $a_9 = 99$, $r = -4$, $a_1 =?$
        \part Calcular $a_{23}$ sabiendo que $a_6=-2$ y $r =8$
        \part En una sucesión de los múltiplos de 9 mayores que 20, ¿Cuál es el término que ocupa el lugar 65?
    \end{parts}

\question En una Sucesión Geométrica, $a_1 = \frac{1}{2}$ y $r = \frac{3}{4}$. Hallar:
    \begin{parts}
        \part El término general $a_n$
        \part Los 7 primeros términos usando el término general
        \part La suma de los 16 primeros términos de la sucesión.
    \end{parts}

\question Resolver justificando cada paso.
    \begin{parts}
        \part $-4 - 3 \cdot 2^{2x} = -4^2$
        \part $2^x - 72 + 4^x = 0$
        \part $\log_{125} x - \log_5 x = -2$
        \part $\log_2 x + \log_2 (-2x + 9) = 2$
    \end{parts}

\question Analizar y graficar la siguiente función racional. Indicar: dominio, imagen, asíntotas, raíces y ordenada al origen. 
    \[f(x) = \frac{3x}{4-2x}\]

\question Analizar y graficar la siguiente función polinómica. Indicar: grado de la función, raíces y orden de multiplicidad de cada una, ordenada al origen, C\textsuperscript{+} y C\textsuperscript{-}.
    \[f(x) = 4(x-3)^2 (x-5) \left( x-\frac{7}{2} \right)^3 \left( x+\frac{1}{2} \right)^2\]

\question Graficar la elipse de ecuación:  $$ \dfrac{(x+1)^2}{9} + \dfrac{(y-3)^2}{25} = 1$$ 
    
    Indicar: centro, vértices y excentricidad.

\end{questions}

\end{document}